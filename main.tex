\documentclass[output=inprep,
  nonflat,
  modfonts,
%  colorlinks,
  showindex,
  draftmode
]{langsci/langscibook}\usepackage[]{graphicx}\usepackage[]{color}
%% maxwidth is the original width if it is less than linewidth
%% otherwise use linewidth (to make sure the graphics do not exceed the margin)
\makeatletter
\def\maxwidth{ %
  \ifdim\Gin@nat@width>\linewidth
    \linewidth
  \else
    \Gin@nat@width
  \fi
}
\makeatother

\usepackage{Sweavel}


  
\title{Many things many linguists should know about the creation, evaluation, and use of corpora*}
\subtitle{* But sometimes don’t bother to ask}
\BackTitle{Creation, evaluation, and use of corpora}
\BackBody{Change your blurb in localmetadata.tex}
\dedication{In memory of Adam Kilgarriff.}
\typesetter{Roland Schäfer}
%\proofreader{Change proofreaders in localmetadata.tex}
\author{Felix Bildhauer and Roland Schäfer}
% \BookDOI{}
\renewcommand{\lsISBNdigital}{000-0-000000-00-0}
\renewcommand{\lsISBNhardcover}{000-0-000000-00-0}
\renewcommand{\lsISBNsoftcover}{000-0-000000-00-0}
\renewcommand{\lsISBNsoftcoverus}{000-0-000000-00-0}
\renewcommand{\lsSeries}{tbls}
\renewcommand{\lsSeriesNumber}{99}
% \renewcommand{\lsURL}{http://langsci-press.org/catalog/book/000}
  

\usepackage{knitr}

\usepackage{tabularx} 
\usepackage{longtable}

\usepackage{./langsci/styles/langsci-optional}
\usepackage{./langsci/styles/langsci-gb4e}
\usepackage{./langsci/styles/langsci-lgr}
\usepackage{./langsci/styles/langsci-glyphs}
\usepackage{./langsci/styles/langsci-tbls}

\usepackage[english]{babel}

% \usepackage[hang,flushmargin]{footmisc}
% \setlength\footnotemargin{10pt}

\usepackage{listings}

\usepackage{unicode-math}

\usepackage{csquotes}
\usepackage{fontspec}

\usepackage{enumitem}


%% hyphenation points for line breaks
%% Normally, automatic hyphenation in LaTeX is very good
%% If a word is mis-hyphenated, add it to this file
%%
%% add information to TeX file before \begin{document} with:
%% %% hyphenation points for line breaks
%% Normally, automatic hyphenation in LaTeX is very good
%% If a word is mis-hyphenated, add it to this file
%%
%% add information to TeX file before \begin{document} with:
%% %% hyphenation points for line breaks
%% Normally, automatic hyphenation in LaTeX is very good
%% If a word is mis-hyphenated, add it to this file
%%
%% add information to TeX file before \begin{document} with:
%% \include{localhyphenation}
\hyphenation{
affri-ca-te
affri-ca-tes
com-ple-ments
}
\hyphenation{
affri-ca-te
affri-ca-tes
com-ple-ments
}
\hyphenation{
affri-ca-te
affri-ca-tes
com-ple-ments
}
\bibliography{localbibliography} 

\usepackage{lipsum}

\setmonofont{Inconsolatar.ttf}

\definecolor{listingbackground}{gray}{0.95}
\lstdefinestyle{RStyle}{
  language=R,
  basicstyle=\ttfamily\footnotesize,
  keywordstyle=\ttfamily\bfseries\color{lsDarkOrange},
  stringstyle=\ttfamily\color{lsDarkBlue},
  identifierstyle=\ttfamily\color{lsDarkGreenOne},
  commentstyle=\ttfamily\color{lsLightBlue},
  upquote=true,
  breaklines=true,
  backgroundcolor=\color{listingbackground},
  framesep=5mm,
  frame=trlb,
  framerule=0pt,
  linewidth=\dimexpr\textwidth-5mm,
  xleftmargin=5mm
  %numbers=left, numberstyle=\color{lsLightGray}, stepnumber=1, numbersep=10pt
  }
  
\lstset{style=Rstyle}


\begin{document}     

% By LSP.
\renewbibmacro*{index:name}[5]{%
  \usebibmacro{index:entry}{#1}
    {\iffieldundef{usera}{}{\thefield{usera}\actualoperator}\mkbibindexname{#2}{#3}{#4}{#5}}}


% By LSP.
\makeatletter
\def\blx@maxline{77}
\makeatother


% Fix line spacing in list environmens.
\setlist{noitemsep}


% RS abbreviations.
\newcommand{\ie}{i.\,e.,\ }


% Correct hyperref colors which otherwise give you eye cancer.
\hypersetup{
  linkbordercolor  = {white}
  , linkcolor        = {lsMidDarkBlue}
  , anchorcolor      = {lsMidWine}
  , citecolor        = {lsDarkGreenOne}
  , menucolor        = {lsMidDarkBlue}
  , urlcolor         = {lsDarkOrange}
%    , filecolor       = {}
%    , runcolor        = {}
}


% Use a better mono font, ideal for code.
% https://github.com/chrissimpkins/codeface/tree/master/fonts/inconsolata-g
\setmonofont{Inconsolata-g}


% Use a math font that actually works! Requires unicode-math paackage.
% https://github.com/khaledhosny/libertinus
\setmathfont[Scale=MatchUppercase]{libertinusmath-regular.otf}


% Set listing style. knitr uses RStyle style. Which you have to know...
\definecolor{listingbackground}{gray}{0.95}
\lstdefinestyle{RStyle}{
  language=R,
  basicstyle=\ttfamily\footnotesize,
  keywordstyle=\ttfamily\color{lsDarkOrange},
  stringstyle=\ttfamily\color{lsDarkBlue},
  identifierstyle=\ttfamily\color{lsDarkGreenOne},
  commentstyle=\ttfamily\color{lsLightBlue},
  upquote=true,
  breaklines=true,
  backgroundcolor=\color{listingbackground},
  framesep=5mm,
  frame=trlb,
  framerule=0pt,
  linewidth=\dimexpr\textwidth-5mm,
  xleftmargin=5mm
  }
\lstset{style=Rstyle}

 
 
\maketitle                
\frontmatter

\currentpdfbookmark{Contents}{name} 
\tableofcontents
\addchap{Preface}


\addchap{Acknowledgments}


Roland Schäfer's work on this project was funded in part by the \textit{Deutsche Forschungsgemeinschaft} (DFG, personal grant SCHA1916/1-1) through the project \textit{Linguistic Web Characterisation}.

\addchap{Abbreviations and symbols}

\section*{Abbreviations}

\begin{longtable}{p{0.1\textwidth}p{0.9\textwidth}}
  ANOVA & analysis of variance \\
  CDF   & cumulative distribution function \\
  CLT   & central limit theorem \\
  cp.   & ceteris paribus (all other things being equal) \\
  iid.  & independent and identically distributed \\
  LM    & linear model \\
  LMM   & linear mixed model \\
  GLM   & linear mixed model \\
  GLMM  & generalised linear mixed model \\
  PDF   & probability density function \\
  VCOV  & variance-covariance matrix \\
\end{longtable}


\section*{Symbols}

Symbols are overloaded ad-hoc to denote either a (possibly indexed) value such as $s_x=1$ (for ``the population mean of variable $x$ is $1$'') or a function such as $s(x)=1$ where applicable.

\begin{longtable}{p{0.1\textwidth}p{0.9\textwidth}} 

		    &  \textbf{Mathematical symbols} \\

  % Pure symbols
  $x\thicksim D$    & \textit{x follows D} ($x$ a variable, $D$ a distribution) \\
  $\bar{x}$         & sample arithmetic mean of $x$\\
  $\tilde{x}$       & sample median of $x$\\
  $\hat{x}$         & predicted value of $x$\\
 
                    & \\
		    &  \textbf{Letter-like symbols} \\

  % Letter-like.
  $\alpha$          & alpha level \\
  $\alpha_i$        & intercept $i$ \\
  $\beta$           & beta level \\
  $\beta_i$         & first-level coefficient $i$ \\
  $df$              & degrees of freedom \\
  $e$               & Euler constant \\
  $\epsilon$        & observation-level error \\
  $f$               & frequency \\
  $F$               & F statistic (see ANOVA)\\
  $\gamma_i$        & second-level coefficient $i$ \\
  $H$               & Kruskal-Wallis statistic \\
  $H_0$             & null hypothesis \\
  $H_A$             & alternative hypothesis \\
  $M_M$             & main hypothesis \\
  $IQR$             & inter-quartile range \\
  $\mathcal{L}$     & Likelihood \\
  $\mu$             & population mean \\
  $\mu_i$           & mean of modeled effect $i$ \\
  $n$               & sample size \\
  $N$               & population size \\
  $O$               & Odds \\
  $p$               & proportion \\
  $P_i$             & the $i$-th percentile \\
  $Pr$              & probability \\
  $\varphi$         & dispersion parameter \\
  $Q_i$             & $i$-th quartile \\
  $r$               & sample covariance coefficient \\
  $r^2$             & coefficient of determination \\
  $R^2$             & multifactorial coefficient of determination \\
  $\rho$            & population covariance coefficient \\
  $s$               & sample standard deviation of $x$ \\
  $s^2$             & sample variance of $x$ \\ 
  $SE$              & standard error \\
  $SS$              & sum of squares \\
  $\sigma$          & population standard deviation \\
  $\sigma^2$        & population variance \\ 
  $U$               & Mann-Whitney statistic \\
  $\chi^2$          & chi square statistic \\
\end{longtable} 

\vspace{\baselineskip}
\noindent Random distributions are denoted by bold-printed abbreviated names instead of the incoherent symbols sometimes used.
\vspace{\baselineskip}

\begin{longtable}{p{0.1\textwidth}p{0.9\textwidth}}
  $\mathbf{Bern}$   & Bernoulli distribution \\
  $\mathbf{Exp}$    & exponential distribution \\
  $\mathbf{F}$      & $F$ distribution \\
  $\mathbf{Norm}$   & normal (Gaussian) distribution \\
  $\mathbf{t}$      & t distribution \\
  $\mathbf{Unif}$   & uniform distribution \\
  $\mathbf{Chisq}$  & $\chi^2$ distribution \\
\end{longtable} 
 
\mainmatter         




\chapter{Science, data, and statistics}
\label{sec:sciencedataandstatistics}

\cite{MacDonaldGardner2000}

\tblssy[lsLightGray]{glass}{Test SY}{\lipsum[66]}

\tblsli[lsLightGray]{1}{Test LI}{\lipsum[66]}

\tblsfi[lsYellow]{Test FI}{\lipsum[66]}

\tblsfr[lsYellow]{glass}{Test FR}{\lipsum[66]}

\tblsfd{lsLightGray}{1}{Test FD}{\lipsum[66]}

\pagebreak

In this book, code listing are displayed as inline blocks such as the following simple code which simulates t-tests under the null hypothesis in order to demonstrate that all p-values have equal probability under the null.

\begin{Schunk}
\begin{Sinput}
# Set simulation parameters.
nsim  <- 1000
n     <- 100
meen  <- 0
stdev <- 1

# Data structure for results.
sims <- rep(NA, nsim)

# Simulations.
for (i in 1:nsim) {
  a <- rnorm(n, mean = meen, sd = stdev)
  b <- rnorm(n, mean = meen, sd = stdev)
  p <- t.test(a,b)$p.value
  sims[i] <- p
}
\end{Sinput}
\end{Schunk}


\begin{Schunk}
\begin{figure}[H]

{\centering \includegraphics[width=\maxwidth]{/Users/user/Workingcopies/SMIL/figuressimttestscatter-1} 

}

\caption[Scatterplot of p-values]{Scatterplot of p-values.}\label{fig:simttestscatter}
\end{figure}
\end{Schunk}

\begin{Schunk}
\begin{figure}[H]

{\centering \includegraphics[width=\maxwidth]{/Users/user/Workingcopies/SMIL/figuressimttestecdf-1} 

}

\caption[Empirical cumulative density distribution of p-values]{Empirical cumulative density distribution of p-values.}\label{fig:simttestecdf}
\end{figure}
\end{Schunk}

\begin{Schunk}
\begin{figure}[H]

{\centering \includegraphics[width=\maxwidth]{/Users/user/Workingcopies/SMIL/figuressimttesthist-1} 

}

\caption[Histogram of p-values]{Histogram of p-values.}\label{fig:simttesthist}
\end{figure}
\end{Schunk}

See Figure~\ref{fig:simttestecdf} for the cumulative density of p-values under the null in a series of 1000 t-tests.
This was plotted using the following command.

\begin{Schunk}
\begin{Sinput}
plot(ecdf(sims))
\end{Sinput}
\end{Schunk}


\chapter{Describing data}
\label{sec:describingdata}


\chapter{Visualising data}
\label{sec:visualisingdata}


\chapter{Tests}
\label{sec:tests}


\chapter{Models}
\label{sec:models}


\chapter{Generalised models}
\label{sec:generalisedmodels}


\chapter{Mixed models}
\label{sec:mixedmodels}




\chapter{Where to go from here?}
\label{sec:wheretogofromhere}

% \is{some term| see {some other term}}
\il{some language| see {some other language}}
\issa{some term with pages}{some other term also of interest}
\ilsa{some language with pages}{some other lect also of interest} 
\backmatter
\phantomsection%this allows hyperlink in ToC to work
\printbibliography[heading=references] 
\cleardoublepage

\phantomsection 
\addcontentsline{toc}{chapter}{Index} 
\addcontentsline{toc}{section}{Name index}
\ohead{Name index} 
\printindex 
\cleardoublepage
  
\phantomsection 
\addcontentsline{toc}{section}{Language index}
\ohead{Language index} 
\printindex[lan] 
\cleardoublepage
  
\phantomsection 
\addcontentsline{toc}{section}{Subject index}
\ohead{Subject index} 
\printindex[sbj]
\ohead{} 
 
\end{document} 

